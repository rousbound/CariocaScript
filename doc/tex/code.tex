\documentclass{article}
\usepackage{listings}
\usepackage{color}
\usepackage[margin=0.8in]{geometry}

\usepackage{tikz}
\usepackage[portuguese]{babel}
\usepackage[utf8]{inputenc}
\usepackage{amsmath}
\usepackage{amssymb}

\definecolor{dkgreen}{rgb}{0,0.6,0}
\definecolor{gray}{rgb}{0.5,0.5,0.5}
\definecolor{mauve}{rgb}{0.58,0,0.82}

\lstset{frame=tb,
  language=C,
  aboveskip=1mm,
  belowskip=1mm,
  showstringspaces=false,
  columns=flexible,
  basicstyle={\small\ttfamily},
  numbers=none,
  numberstyle=\tiny\color{gray},
  keywordstyle=\color{blue},
  commentstyle=\color{dkgreen},
  stringstyle=\color{mauve},
  breaklines=true,
  breakatwhitespace=true,
  tabsize=3
}

\title{\vspace{-2cm}CariocaScript}
\author{Geraldo Luiz de Carvalho Pereira Junior}
\date{}

\begin{document}
\maketitle


\begin{itemize}
	\item{\textbf{Curso:} INF 1022}
	\item{\textbf{Semestre:} 2019.2}
	\item{\textbf{Professor:} Edward Hermann Haeusler}
	%\item{\textbf{Grupo:} }
	%\begin{itemize}
		%\item{Geraldo Luiz de Carvalho Pereira Junior}
	%\end{itemize}
\end{itemize}

\section{Rodando o programa}

Para rodar o programa é só executar o arquivo "Coe" passando o argumento de um programa com extensão ".cara":

\begin{lstlisting}
$ ./Coe tests/t3.cara
\end{lstlisting}

\section{Principais mudanças}

A gramática da CariocaScript foi feita a partir da Provol-One:

\bigskip
\noindent
program $\to$ ENTRADA varlist SAIDA varlist cmds FIM\\
varlist $\to$ id varlist $|$ id\\
cmds $\to$ cmd cmds $|$ cmd\\
cmd $\to$ FACA id VEZES cmds FIM\\
cmd $\to$ ENQUANTO id FACA cmds FIM\\
cmd $\to$ SE id ENTAO cmds SENAO cmds $|$ SE id ENTAO cmds\\
cmd $\to$ id = id $|$ INC(id) $|$ ZERA(id)

\bigskip
\noindent
Que acabou se tornando:

\bigskip
\noindent
program $\to$ CHEGAMAIS input NAMORAL varlist cmds VALEU\\
input $\to$ varlist\\
varlist $\to$ id varlist $|$ id\\
cmds $\to$ cmd cmds $|$ cmd\\
cmd $\to$ MARCA id RAPIDAO cmds VALEU\\
cmd $\to$ ENQUANTO id FACA cmds VALEU\\
cmd $\to$ SEPA id TA_LGD cmds SENAO cmds $|$ SE id VALEU cmds\\
cmd $\to$ id = id $|$ id++ $|$ id-- $|$ RELAXOU(id) $|$ FALATU(id) $|$ id += id $|$ id -= id

\section{Geração de Código}

\subsection{Contagem de Labels e fluxo de controle}
A geração de código da CariocaScript se dá em puro assembly. A maior
dificuldade em montar um bloco de código em assembly é na tradução das
instruções de controle de fluxo. O sistema é na verdade muito simples na
maioria dos casos, apenas é necessário adicionar +1 a contagem das labels a
cada cmd executado que tudo passa a encaixar perfeitamente(até então). Já com
comandos como \textit{ENQUANTO id FACA cmds VALEU e SEPA id TA\_LGD cmds SENAO cmds},
 é necessário fazer uma pequena alteração no sistema de contagem
das labels, pois a instrução exige duas labels diferentes. Nas demais, como
\textit{SEPA id TA\_LGD cmds VALEU} só é necessário referenciar a mesma label.

\subsection{Administração de variáveis}
Outra curiosidade é a administração de variáveis, que se dá através de alocação
na pilha através de um cálculo simples. Todas as operações são feitas
referenciando a pilha, apenas usando um registrador callee-saved quando
necessário para fazer operações que são ilegais para dois endereços
simultâneos, como \textit{movl -8(\%rbp), -16(\%rbp)}(que corresponde a instrução id = id
, que é contornada por:
\begin{lstlisting}
movl -8(%rbp), %r12d
movl %r12d, -16(%rbp)
\end{lstlisting}

\section{Exemplos}

Um programa como:

\begin{lstlisting}
CHEGAMAIS X,Y,Z NAMORAL
Z=X
MARCA X RAPIDAO
	Z += Y
VALEU
FALATU(X)
FALATU(Y)
FALATU(Z)
VALEU
\end{lstlisting}

\noindent
Resulta no seguinte código de máquina:


\begin{lstlisting}
.globl  cariocaScript
 Si:  .string "Meu Brother: "

 Sii:  .string "%d"

 Nl:  .string "\n"

 Sf:  .string "Meu Parcerasso:%d\n"

 cariocaScript:
   pushq %rbp
   movq %rsp,%rbp
   subq $32, %rsp

   movq $Si, %rdi
   call printf
   movq $Sii, %rdi
   leaq -8(%rbp), %rsi
   call scanf

   movq $Sii, %rdi
   leaq -16(%rbp), %rsi
   call scanf

   movq $Sii, %rdi
   leaq -24(%rbp), %rsi
   call scanf

   movq $Nl, %rdi
   call printf

   movl -8(%rbp), %r12d 
   movl %r12d, -24(%rbp) 
   movl $0,%r13d
L2:
   addl $1,%r13d 
   movl -16(%rbp), %r12d 
   addl %r12d, -24(%rbp) 
   cmpl %r13d,-8(%rbp)
   jne L2

   movq $Sf, %rdi
   movl -8(%rbp), %esi
   call printf

	 movq $Sf, %rdi
   movl -16(%rbp), %esi
   call printf

	 movq $Sf, %rdi
   movl -24(%rbp), %esi
   call printf

	 movq %rbp, %rsp
   popq %rbp
   ret
\end{lstlisting}


\end{document}
